% 預設的字型大小、紙張大小,文件類型為 article、簡報通常為 beamer, 寫書:book, 長篇報告:report, 信件:letter
\documentclass[12pt, letterpaper]{article}
% 此文件的編碼,建議使用 UTF-8
\usepackage[utf8]{inputenc}

% 中文與字型設定
\usepackage{xeCJK}
    % 設定中文字型
\setCJKmainfont[ Path = ./fonts/]{BiauKai.ttf}
    % 設定英文字型
\setmainfont[ Path = ./fonts/]{Times New Roman.ttf}

% 引入註解套件
\usepackage{comment}

% 文章標題設定
\title{\LaTeX 光速入門}
\author{呂栢頤 \thanks{感謝 ShareLaTeX team 與炎龍老師的文獻}}
\date{\today}

\begin{document}

% 建立標題
\begin{titlepage}
\maketitle
\end{titlepage}

% 摘要
\begin{abstract}
大家好,我是呂栢頤,目前任職於中華電信研究院。這份教學將帶大家很快的了解 \LaTeX 的一些基本操作,並提供一個簡易的模板讓大家使用!
\end{abstract}

% 註解
\begin{comment}
這裡的文字不會顯示在文章中, \LaTeX 裡註解的方法除了 \% 外,這也是一個常用的多行註解方式,需要搭配 \textit{comment} 套件。
\end{comment}

\section{引言}
這裡,讓我很快速的按照 \textit{SHARELATEX} 教學文件讓大家瞭解 \LaTeX 一些性質。

文章的「段落」與「空行」:

『因為在推文看到 "迪士尼告死小老百姓"的事情,決定打一篇。迪士尼超級注重版權,在商業的層面上啦。你沒經過迪士尼授權隨便印個米老鼠在內褲上賣給小女孩,這個會被告。你沒經過迪士尼授權在P網畫了18禁的高飛圖(很顯然也不會授權給你),有點危險,但真的有人被告過嗎?目前還沒有。看看deviantart 看看 PIXIV,一堆都活的好好的啊...』

從這裡開始是第二段,上面那篇是從 ptt C 洽版的 \textit{OlaOlaOlaOla} 所發的文章。第二段在 M\$ Word 裡面其實就是 `enter' 指令,有時候我們不想分段時就會用到 `shift-enter',這在 \LaTeX 裡面只要輸入 $\backslash$ $\backslash$ 就好囉,像這樣 \\
有看到了嗎?新的一行就這樣展開惹 $\textasciitilde$ $\textasciitilde$ 更進一步,在 Word 裡面會使用 `ctrl-enter' 換到下一面,這個也可以利用 $\backslash$ newpage 辦到,像這樣 \newpage

新的一頁就出來囉!

另外有些符號是不能直接使用的,我們通常稱做「功能符號」:

\begin{itemize}
\item \#:套件參數指令,我們可以像是定義函數一樣定義新的指令,詳見 $\backslash$command 的操作
\item \$:啟用數學模式,最吸引人的功能!詳見之後的章節
\item \%:註解,我剛剛其實已經偷偷用了喔 $\textasciicircum$ \_ $\textasciicircum$
\item $\textasciicircum$:數學模式中的上標(指數)位置
\item \&:表格的欄位切割指令,詳見之後的章節
\item \_:數學模式中的下標(底數)位置
\item \{ \}:使用指令需要分隔的區域,指令的括號
\item $\textasciitilde$:小空格
\item $\backslash$:指令開始時所下的參數
\end{itemize}

很多人害怕 \LaTeX 是害怕他的指令與操作模式,但是其實沒那麼難,我個人還覺得一直打字不需要滑鼠是件幸福的事呢。

下一章是本文件最可怕(?)的地方,就是 \LaTeX 的安裝,其實現在已經變得很簡單了,只是要怎麼從零開始打造 \LaTeX 文件呢?讓我們繼續看下去...

\section{\LaTeX 的安裝與環境}

\end{document}
